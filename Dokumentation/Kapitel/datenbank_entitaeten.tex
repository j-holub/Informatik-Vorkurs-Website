
\documentclass[Info_VK_Website_Dokumentation.tex]{subfiles}

\begin{document}
	
\chapter{Datenbank Entitäten}

\section{User}

Der User repräsentiert einen Erstie/Studenten, der am Informatik Vorkurs teilnimmt.

Jeder Erstie/Student benötigt einen Account, der mit solch einem User verknüpft ist, um sich einloggen zu können. Damit hat er Zugang zu all den Informationen und Funktionen der Website, wie z.B. die Installationsanleitungen und Roboter hochladen.

Das Userobjekt speichert auch einige Informationen über den User, wie Kontaktinformationen und Ähnliches.

\begin{table}[H]
\centering
\begin{tabular}{r | c | l}
\textbf{Name} & \textbf{Beschreibung} & \textbf{Typ} \\
\hline
\hline
\_id  & Eindeutige Id & String \\
\hline
email   & Email Adresse & String \\
\hline
userPic & Profilbild   & Bild oder Gravatar \\
\end{tabular}
\end{table}


\section{Roboter}

Die Ersties können die Roboter, die sie programmiert haben auf der Website hochladen, um diese dann zu einem Turnier anzumelden.


\begin{table}[H]
\centering
\begin{tabular}{ r | c | l }
\textbf{Name} & \textbf{Beschreibung} & \textbf{Typ} \\
\hline
\hline
\_id       & Eindeutige Id & String \\
\hline
name     & Name          & String \\
\hline
description & Beschreibung des Roboters & String \\
\hline
dateUploaded & Datum an dem der Roboter hochgeladen wurde & date \\ 
\hline
belongsTo & Id des Users, dem dieser Roboter zugehörig ist & String \\
\end{tabular}
\end{table}



\section{Turnier} 

Wenn ein Turnier offen ist, können User ihre Roboter dafür anmelden. Die Idee dahinter ist, dass man dann nur noch auf einen Button drücken muss und dann alle Roboter für das Turnier direkt als Download bekommt.


\begin{table}[H]
\centering
\begin{tabular}{ r | c | l}
\textbf{Name} & \textbf{Beschreibung} & \textbf{Typ} \\
\hline
\hline
\_id       & Eindeutige Id & String \\
\hline
date  & Datum an dem das Turnier statt findet &  date \\
\hline
participantList & teilnehmende Roboter  & wird noch bestimmt \\ 
\end{tabular}
\end{table} 

\end{document}