\documentclass[Info_VK_Website_Dokumentation.tex]{subfiles} 

\begin{document}
	
\chapter{Technologien}

\section{Meteor} 

Die Website ist größtenteils im Meteor Framework geschrieben. Dieses macht es einem sehr leicht Echtzeit Websites zu bauen.

Das Framework nutzt größtenteils JavaScript.

Als Datenbank kommt bei Meteor MongoDb zum einsatz, ein \emph{Non-relational} Datenbankmodell.

\begin{itemize}
 	\item Meteor Framework \url{http://meteor.com} \\
 	\item Meteor Tips - sehr gutes Tutorial \url{http://meteortips.com} \\
 	\item MongoDb \url{https://www.mongodb.org} \\ 
\end{itemize} 

\subsection{Benutzte Packages}

Packages können mit dem Befehel \emph{meteor add <package>} installiert werden.

\begin{itemize}
 	\item \textbf{iron:router} - Zum handeln von URL routen \\
 	\item \textbf{accounts-password} - User handling \\
 	\item \textbf{cfs:standard-packages} - Basispaket für FileUpload handling \\
 	\item \textbf{cfs:filesystem} - Packet um uploads im normalen Dateisystem zu speichern \\
 	\item \textbf{cfs:ejson-file} - Zur Dataireferenzierung in Collections \\
 	\item \textbf{ejson} - Extended Json Library \\
 	\item \textbf{check} - Serverseitiges checken \\
 	\item \textbf{utilities:avatar} - User Avatars \\
 \end{itemize} 

\section{SASS}

SASS ist eine art Precompiler der CSS Dateien ausspuckt.

\begin{itemize}
	\item SASS \url{http://sass-lang.com} 
\end{itemize}




\end{document}