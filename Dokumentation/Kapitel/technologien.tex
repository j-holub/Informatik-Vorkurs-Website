\documentclass[Info_VK_Website_Dokumentation.tex]{subfiles} 

\begin{document}
	
\chapter{Technologien}



\section{Meteor} 

Die Website ist größtenteils im Meteor Framework geschrieben. Dieses macht es einem sehr leicht Echtzeit Websites zu bauen.

Das Framework nutzt größtenteils JavaScript.

Als Datenbank kommt bei Meteor MongoDb zum einsatz, ein \emph{Non-relational} Datenbankmodell.

\begin{itemize}
 	\item Meteor Framework \url{http://meteor.com} \\
 	\item Meteor Tips - sehr gutes Tutorial \url{http://meteortips.com} \\
 	\item MongoDb \url{https://www.mongodb.org} \\ 
\end{itemize} 

\subsection{Benutzte Packages}

Packages können mit dem Befehel \emph{meteor add <package>} installiert werden.

\begin{itemize}
 	\item \textbf{iron:router}\\
 	Zum handeln von URL routen \\
 	\small \url{https://github.com/iron-meteor/iron-router} 

 	\item \textbf{accounts-password}\\
 	User handling \\
 	\small \url{https://www.meteor.com/accounts} 

 	\item \textbf{aldeed-collection2}\\
 	Damit kann man Datenbankschemas vorgeben. Hilfreich bei der Validierung \\
 	\small \url{https://github.com/aldeed/meteor-collection2} 

 	\item \textbf{cfs:standard-packages}\\
 	Basispaket für FileUpload handling\\
 	\small \url{https://github.com/CollectionFS/Meteor-CollectionFS} 

 	\item \textbf{cfs:filesystem}\\
 	Packet um uploads im normalen Dateisystem zu speichern\\
 	\small \url{https://github.com/CollectionFS/Meteor-CollectionFS} 

 	\item \textbf{cfs:ejson-file}\\
 	Zur Dataireferenzierung in Collections\\
 	\small \url{https://github.com/CollectionFS/Meteor-CollectionFS} \\ 

 	\item \textbf{ejson}\\
 	Extended Json Library\\
 	\small \url{https://atmospherejs.com/meteor/ejson}

 	\item \textbf{check}\\
 	Serverseitiges checken\\
 	\small \url{https://atmospherejs.com/meteor/check} 

 	\item \textbf{utilities:avatar}\\
 	User Avatars\\
 	\small \url{https://github.com/meteor-utilities/avatar} 

 	\item \textbf{meteorhacks:search-source}\\
 	Suchstruktur \\
 	\small \url{https://github.com/meteorhacks/search-source}

 	\item \textbf{fortawesome:fontawesome}\\
 	FontAwesome Icon Paket
 	\small \url{http://fontawesome.io} 

 	\item \textbf{seba:minfiers-autoprefixer}\\
 	Prefixt automatissch die CSS Befehlele

 	\item \textbf{fourseven:scss}\\
 	Kompiliert automatisch alle Sass Dateiei \\
 	\small \url{https://github.com/fourseven/meteor-scss} 

 	\item \textbf{chrismbeckett:toastr}\\
 	Eine JQuery Bilbliothek für Toasts \\
 	\small \url{https://github.com/CodeSeven/toastr} 

 	\item \textbf{natestrauser:jquery-scrollto}\\
 	JQuery Bilbiothek zum automatischen Scrollen \\
 	\small \url{https://github.com/flesler/jquery.scrollTo} 

 \end{itemize} 

\subsection{Entfernte Pakete}

\begin{itemize}
 	\item \textbf{autopublish}\\
 	Gibt automatisch alle Daten für jeden User frei
 	\item \textbf{insecure}\\
 	Lässt Datenbankoperationen auf dem Client zu
 	\item \textbf{standard-minifiers}\\
 	Der Standart Minifier, der alle CSS Dateien in eine packt
 \end{itemize} 



\section{SASS}

SASS ist eine art Precompiler der CSS Dateien ausspuckt.

\begin{itemize}
	\item SASS \url{http://sass-lang.com} 
\end{itemize}

\section{Grid}

Das Projekt verwendet ein CSS Grid zum besseren Layouten der Seiten. Ebenso wird das Reponsive Design davon unterstützt. Benutzt wird hier das Grid aus dem MaterialzeCSS Framework.

\begin{itemize}
 	\item \url{http://materializecss.com/grid.html} 
\end{itemize} 




\end{document}